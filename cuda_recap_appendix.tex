\section{Appendix}
\subsection*{Improving with primitive operations}\label{App.:Primitive operations}
The modulo operation \verb|%| is expensive for an arithmetic operation. One could replace it 
by something much easier for the GPU architecture to execute. In this case, with a bit of 
knowledge of binary number representation, we can verify that \verb|if( (id_x%(stride*2) ) == 0)| is 
equivalent to \verb|if( (id_x&(stride*2 -1) ) == 0)|. It would impossible for me to come up with this small 
optimization on my own. Maybe for CS majors, it is evident.

\subsection*{Warp level filtering and syncronization}
Let's have a look at three functions for warp-local syncronizations - \verb|__all_sync()|, \verb|__any_sync()| and \verb|__ballot_sync()|.
The aim of the piece of code is to show the working principle of these functions, with made-up examples.
\begin{code}
\inputminted[frame=single, framesep=1mm, breaklines, linenos=true]{cuda}{cucodes/all_any_sync.cu}

\label{app:all_any_ballot}
\end{code}

The output of this program on my computer is given by: 
\begin{code}
\inputminted[breaklines, frame=single, tabsize=1]{zsh}{cucodes/output_all_any.sh}
\end{code}
The code is quite straightforward to understand, but let's break it down. The program contains 
2 independent functions - one to demonstate the \verb|__all_sync()| and \verb|__any_sync()| functions, 
and one the \verb|__ballot_sync()|. Let's start with the first one.


The functions runs on 32 threads, that is, one warp. This is not surprising, as we're dealing with 
warp-local primitives, which we want to illustrate. The example consists of applying the 2 functions of 
interest to determine, whether there are functions having an even ID. Which is, of course true, as indices in 
a single lane go from $0$ to $31$. So first we're retrieving the index of the thread and constructing the variable, 
\verb|id_is_even|, which will be \verb|true| if the id is even, and \verb|false| otherwise. It is clear, that there will be 16
threads, in which it evaluates to \verb|true|, and 16, where it evaluates to \verb|false|. Next, we're asking the CUDA 
runtime, to give us the mask with active threads. In our case, all of the launched threads will be active. Thus the \verb|mask| 
evaluates to \verb|-1| or \verb|0xffffffff| in hex. Finally, were using the two functions of interest. From each thread, 
the function will "look into other's threads" and evaluate the predicate. From that it is clear, that the return values from 
the specific function will be the same in all threads. After applying the functions, we're synchronizing threads, which is, maybe, 
unnecessary. Finally we're printing the results of the 2 evaluations (we're doing it only once, as the results are the same in all threads within the warp). The result is not surprising. Indeed, the function \verb|__all_sync()| will return false, as \textbf{not all thread's 
id is even} (there are also odd id's). The function \verb|__any_sync()|, on the other hand, returns true, as there is at least one 
function, that fulfills the condition of having an even index.


The second function is similar, yet more \textit{more precise}. That is, we can retrieve the precise index of the thread within the lane, 
where the predicate has been evaluated to \verb|true|. In this case, the predicate will be true if the thread's lane index is equal to the 
arbitrary chosen value $12$. There will be only one thread, which fulfills this condition, namely - the one with the lane id $12$. Thus, 
similarly to the previous example, we retrieve the id and compute the predicate. Then retrieve the mask for active threads, and 
pass it, together with the predicate, to the \verb|__ballot_sync()|. Then printing the result (we're doing it only once, as the 
result is the same for all the threads within the warp). The function that prints the results simply accesses integer's 
consecutive bits. That is, in order to obtain integer's $n$ $i$'th bit, one writes \verb|(n & (1 << i) ) >> 1|. The result is expected: 
all bits are $0$, except the 12'th in the sequence.












